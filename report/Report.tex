% VDE Template for EUSAR Papers
% Provided by Barbara Lang und Siegmar Lampe
% University of Bremen, January 2002
% English version by Jens Fischer
% German Aerospace Center (DLR), December 2005
% Additional modifications by Matthias Wei{\ss}
% FGAN, January 2009

%-----------------------------------------------------------------------------
% Type of publication
\documentclass[a4paper,10pt]{article}
%-----------------------------------------------------------------------------
% Other packets: Most packets may be downloaded from www.dante.de and
% "tcilatex.tex" can be found at (December 2005):
% http://www.mackichan.com/techtalk/v30/UsingFloat.htm
% Not all packets are necessarily needed:
\usepackage[T1]{fontenc}
\usepackage[latin1]{inputenc}
%\usepackage{ngerman} % in german language if required
\usepackage[nooneline,bf]{caption} % Figure descriptions from left margin
\usepackage{times}
\usepackage{multicol}
\usepackage{amsmath}
\usepackage{amssymb}
\usepackage{graphicx}
\usepackage{epsfig}
\usepackage{listings}
\usepackage{color}
\usepackage{nameref}
\usepackage{url}
\usepackage{mdframed}

\input{tcilatex}
%-------------------------------------------------------------------------------
% Page Setup
\textheight24cm \textwidth17cm \columnsep6mm
\oddsidemargin-5mm                 % depending on print drivers!
\evensidemargin-5mm                % required margin size: 2cm
\headheight0cm \headsep0cm \topmargin0cm \parindent0cm
\pagestyle{empty}                  % delete footer and header
%------------------------------------------------------------------------------
% Environment definitions
\newenvironment*{mytitle}{\begin{LARGE}\bf}{\end{LARGE}\\}%
\newenvironment*{mysubtitle}{\bf}{\\[1.5ex]}%
\newenvironment*{myabstract}{\begin{Large}\bf}{\end{Large}\\[2.5ex]}%
%-------------------------------------------------------------------------------
% Using Pictures and tables:
% - Instead "table" write "tablehere" without parameters
% - Instead "figure" write "figurehere " without parameters
% - Please insert a blank line before and after \begin{figuerhere} ... \end{figurehere}
%
% CAUTION:   The first reference to a figure/table in the text should be formatted fat.
%
%
%\begin{figurehere}
% \centering
% \includegraphics[width=8cm, height=4cm]{./eps/placeholder.eps}
% \caption{Some single-column figure caption.}
% \label{fig:myfigure1}
%\end{figurehere}
%
%\begin{figure*}[t]
%  \centering
% \includegraphics[width=16cm, height=4cm]{./eps/placeholder.eps}
% \caption{Some wide-figure caption.}
% \label{fig:myfigure2}
%\end{figure*}

\makeatletter
\newenvironment{tablehere}{\def\@captype{table}}{}
\newenvironment{figurehere}{\def\@captype{figure}\vspace{2ex}}{\vspace{2ex}}
\makeatother

\newenvironment{packed_items}{
\begin{itemize}
  \setlength{\itemsep}{3pt}
  \setlength{\parskip}{0pt}
  \setlength{\parsep}{0pt}
}{\end{itemize}}

\definecolor{lightgreen}{rgb}{0.0,0.7,0.0}
\definecolor{lightblue}{rgb}{0.0,0.0,0.7}
\definecolor{lightgrey}{rgb}{0.6,0.6,0.6}

\newcommand{\icc}{I\textsuperscript{2}C }


%%%%%%%%%%%%%%%%%%%%%%%%%%%%%%%%%%%%%%%%%%%%%%%%%%%%%%%%%%%%%%%%%%%%%%%%%%%%%%%%
\begin{document}

% Please use capital letters in the beginning of important words as for example
\begin{mytitle}\icc on a Linux based embedded system\end{mytitle}
\begin{mysubtitle}
Design of a bus driver and a client driver for the Nomadik NHK8815 platform
\end{mysubtitle}
%
% Please do not insert a line here
%
\\
Ghiringhelli Fabrizio\\
Matr. 753368, (fabrizio.ghiringhelli@mail.polimi.it)\\
\hspace{10ex}
\begin{flushright}
\emph{Report for the master course of Embedded Systems}\\
\emph{Reviser: PhD. Patrick Bellasi (bellasi@elet.polimi.it)}
\end{flushright}

Received: September, ?? 2012\\
\hspace{10ex}

\begin{myabstract} Abstract \end{myabstract}
TODO

\vspace{4ex}	% Please do not remove or reduce this space here.
\begin{multicols}{2}

%%%%%%%%%%%%%%%%%%%%%%%%%%%%%%%%%%%%%%%%%%%%%%%%%%%%%%%%%%%%%%%%%%%%%%%%%%%%%%%
\section{Introduction}
Today Linux is the operating system choice for many computer systems which
include, not only desktop and server supercomputers, but also a wide range of 
special-purpose electronic devices known as embedded systems.
An embedded system is specifically designed to perform a set of designated
activities, and it generally uses custom, heterogeneous processors. This makes
Linux a flexible operating system capable of running on a variety of
architectures, such as ARM, PowerPC, MIPS, SPARC, x86, and many others.

However, this flexibility doesn't come for free. While it's true that the
Linux highly modular architecture facilitates the porting phase, still a
lot of efforts are required to build new kernel components to fully
support the target platform.

A big part of these efforts are in developing the low-level interfaces
commonly referred to as \emph{device drivers}.
A device driver (driver for short) is a piece of software designed to direct
control a specific hardware resource using an hardware-independent well defined
interface.

This paper details the design of two \icc drivers for the Nomadik NHK8815
platform: a client driver for an on-board inertial sensor, presented in
section \ref{sec:i2c_client_driver}, and a bus driver for the Sistem-On-Chip
\icc controller (section \ref{sec:i2c_bus_driver}).

The rest of this section provides an overview of the \icc protocol
(\ref{sec:i2c_protocol_overview}), a brief description of the NHK8815 evaluation
board (\ref{sec:nomadik_nhk8815_platform}), and detail information regarding
the project environment (\ref{sec:project_setup}). Finally, the section
\ref{sec:tools} describes the tools and scripts used throughout the project.



%-------------------------------------------------------------------------------
\subsection{I2C protocol overview}
\label{sec:i2c_protocol_overview}
The Inter-Integrated Circuit, or \icc, is a synchronous master-slave messaging
protocol designed to connect a pool of devices by means of a two-wire bus.
It is a simple and low-bandwidth protocol which allows for short-distance on
board communications, while being extremely modest in its hardware resource
requirements. The original standard specified a standard clock rate of 100KHz.
Later updates to the standard introduced a fast speed of 400KHz and a high speed
of 1.7 or 3.4 MHz.

The \icc bus consists of two bi-directional lines, one line for data (SDA)
and one for clock (SCL), by means of which a single master device can send
informations serially to one ore more slave devices (Figure
\ref{fig:i2c-implementation}).
To prevent any conflict every device hooked up to the bus have its own unique
address. The standard \icc specifies two different addressing schema, 7 and
10 bits, allowing at most 128 and 1024 devices connected at the same time.

\begin{figurehere}
 \centering
 \includegraphics[width=6cm, height=3.8cm]{./figures/i2c-diagram.png}
 \caption{Sample \icc implementation (adapted from \emph{embedded-lab.com}).}
 \label{fig:i2c-implementation}
\end{figurehere}

Each \icc transaction is always initiated by the master which is in charge of
the bus for the entire duration of the transaction, meaning that it controls the
clock and generates the START and STOP sequences. The start and stop sequences
mark the beginning and the end of a transaction and are the only places where
the SDA line is allowed to change while the SCL is high.

All data are transfered one byte at a time. In 7-bit addressing mode, the slave
address occupies the seven most significant bits of the first byte, with the
least significant bit serving as a read/write flag to indicate whether data
will be written to the slave ('0') or data will be read from the slave ('1').
For every byte received, the slave device sends back an acknowledge bit.
Figure \ref{fig:i2c-implementation} shows an example of a typical \icc
transaction.

The \icc protocol supports multiple masters. In a multi-master environment two
or more masters may simultaneously attempt to initiate a data transfer.
In such a scenario, each master must be able to detect a collision and to follow
the arbitration logic that leeds to the election of a winner master. The winner
master can then safely begin its transaction.
The Nomadik NHK8815 platform, like most system designs, operates in a 
single-master environment.

\begin{figurehere}
 \centering
 \includegraphics[width=8cm, height=3.3cm]{./figures/i2c-transaction.jpg}
 \caption{Sample \icc transaction (adapted from \emph{www.ermicro/blog}).}
 \label{fig:i2c-transaction}
\end{figurehere}

\emph{The Linux \icc subsystem}\\[6pt]
The Linux kernel \icc framework consists of a core layer where resides all the
routines and data structures available to bus drivers and client drivers
(figure \ref{fig:linux-i2c-subsystem}).
The core also provides a level of indirection that allows the underlying drivers
to change from one system to another without affecting \icc subsystem that
relies on them.

This philosophy of a core layer and its attendant benefits is an example of how
Linux helps portability. For instance, enabling \icc on a new platform (which is
precisely the objective of this project) requires only to design the
hardware-dependent components, namely the bus driver and the client drivers,
whereas the core layer needs not to be changed.

\begin{figurehere}
 \centering
 \includegraphics[width=7.5cm, height=6cm]{./figures/linux-i2c-subsystem.png}
 \caption{The Linux \icc subsystem (reprinted from \cite{venkateswaran2008eldd},
 		p. 236).}
 \label{fig:linux-i2c-subsystem}
\end{figurehere}



%-------------------------------------------------------------------------------
\subsection{The Nomadik NHK8815 platform}
\label{sec:nomadik_nhk8815_platform}
The NHK8815 is a full-featured evaluation board for the ST Microelectronics
Nomadik STn8815 (figure \ref{fig:nomadik-nhk8815}). It is a fan-less embedded
computer equipped with a wide range of peripheral devices including USB, UART,
LAN, WLAN, Bluetooth, FM radio, SIM card reader, SD/MMC card reader, color LCD
with touch screen controller, key-pad, video encoder, audio codec, FM radio,
three-axis accelerometer, etc..

\begin{figurehere}
 \centering
 \includegraphics[width=8cm, height=6.35cm]{./figures/nomadik-nhk8815.jpg}
 \caption{The Nomadik NHK8815 evaluation board.}
 \label{fig:nomadik-nhk8815}
\end{figurehere}

The core of the board is the Nomadik STn8815 multimedia application processor.
The STn8815 is a \emph{System On Chip} that combines an ARM9 core up to 332MHz
with level-two cache to audio, video, imaging and graphics accelerators.

Among other interfaces, the STn8815 integrates two \icc high speed controllers
that support the physical and data-link layer of the \icc protocol.
Below is a list of the \icc controllers' main features:
\begin{packed_items}
	\item Slave transmitter/receiver and master transmitter/receiver modes.
	\item Standard (100kHz), fast (400kHz) and high-speed (3,4MHz) baud rates.
	\item 7-bit or 10-bit addressing.
	\item Compliance with \icc standards.
\end{packed_items}

In a traditional \icc bus topology such as the one in figure
\ref{fig:i2c-implementation}, the STn8815 \icc embedded controller plays the
role of the master. In order for the controller to access the \icc bus, a
corresponding driver must be implemented and registered with the \icc
subsystem. This step is covered in section \ref{sec:i2c_bus_driver},
"\nameref{sec:i2c_bus_driver}."

Because on the NHK8815 evaluation board the majority of the on-board peripheral
devices is connected to the STn8815 \icc bus 0, in this project I focus the
attention on this bus rather than on bus 1.
The table \ref{table:i2c-device-addresses} reports the list of devices on 
the \icc bus 0, with their 7-bit address.
Just like for the case of the bus driver, in order for each slave device
to function a corresponding driver must be registred to the \icc subsystem.
My choice for this project was the LIS3LV02D, a MEMS inertial sensor mapped
at the address 1D on the Nomadik board.
The design of the LIS3LV02D driver is detailed in section
\ref{sec:i2c_client_driver}, "\nameref{sec:i2c_client_driver}."\\[6pt]

\begin{tablehere}
	\centering
	\renewcommand{\arraystretch}{1.2}	
	\begin{tabular}{l l c}
		\hline
		Device type & Description & 7-bit address \\
		\hline
		STw5095 & Stereo audio codec & 1A \\
		LIS3LV02DL & Three-Axes accelerometer & 1D \\
		STw8009 & Video digital encoder	 & 21 \\
		TDA8023TT & Smart Card interface & 22,23 \\
		STw4811 & Power management & 2D \\
		STMPE2401 & Port expanders & 43,44 \\
		TSC2003IPW & Touch screen controller & 48 \\
		STw4102 & Battery charger & 70 \\
		\hline
	\end{tabular}
	\caption{NHK8815 \icc device address map.}
	\label{table:i2c-device-addresses}
\end{tablehere}



%-------------------------------------------------------------------------------
\subsection{Project setup}
\label{sec:project_setup}
% TODO:
% x project directory layout
% x environmental variables
% x explain why starting from original linux torvalds (no ST linux)
% - bbfs

The first step to the project is to get the Linux kernel source and to configure
it for the target board. Instead of  using a vendor supplied kernel such as like
STLinux, my personal strategy for this project was to work with the official
kernel from \emph{www.kernel.org}.
This decision was motivated by the need to keep the development process as neat
as possible, and to avoid performing tasks that are not strictly concerned with
the project.

It should be noted that the official kernel offers only limited support for the
Nomadik  NHK8815 platform.
This includes basic definitions and register addresses, interrupt management,
resource definitions and timers. The \icc interface can be used through the
generic drivers \emph{i2c-gpio} and \emph{i2c-algobit}, but no one of the NHK8815
\icc slave devices is provided.

I cloned the git repository at kernel.org on a local machine folder
located \underline{outside} the project workspace. From the tag \emph{3.3}
I created a new branch called \emph{i2cdevel}, where to commit the kernel updates.
I generated one patch per commit. All patches are numbered in sequence and
stored in a folder \underline{inside} the project workspace called
\emph{linux-3.3.0-patches}.

One benefit of this approach is that the project workspace, which is a git
repository as required, also holds all the data needed to obtain the modified
kernel from the official one.
Table \ref{table:project-dir-layout} shows the project workspace layout.
In addition, to ease the project management I defined a set of environmental
variables. A list of them with their definition is presented in table
\ref{table:project-env-variables}.

Finally, one note about the root filesystem. It was build using BusyBox
(version 1.18.4). More precisely, I used a tool called \textbf{bbfs 1.3}, 
developed by \emph{Alessandro Rubini} and released under GNU GPL license
(see \cite{bbfs1.3} for details).\\

\begin{tablehere}
	\centering
	\renewcommand{\arraystretch}{1.2}	
	\begin{tabular}{l p{5cm}}
		\hline
		Variable & Decription \\
		\hline
		\texttt{PRJROOT} & The project root \\
		\texttt{KERNELDIR} & The root directory of the kernel source tree \\
		\texttt{PATCHESDIR} & The \emph{linux-3.3.0-patches} directory \\
		\texttt{ROOTFS} & The directory that holds the root filesystem \\
		\texttt{MODULESDIR} & The directory for temporary holding of the
			loadable driver modules \\
		\hline
	\end{tabular}
	\caption{Project environmental variables.}
	\label{table:project-env-variables}
\end{tablehere}


\begin{table*}
	\centering
	\renewcommand{\arraystretch}{1.2}	
	\begin{tabular}{p{5cm} p{11cm}}
		\hline
		Directory & Content \\
		\hline
		\texttt{linux-3.3.0-patches} & The patches to the Linux kernel and the 
			kernel configuration file \emph{.config}\\
		\texttt{drivers} & The bus driver and slave driver (one subfolder per
			each driver) \\
		\texttt{report} & The \LaTeX sources of this paper \\
		\texttt{script} & Shell scripts (stored in the root folder of the target
			filesystem) \\
		\texttt{tools} & User-space programs to evaluate the drivers (stored in
			the root folder of the target filesystem) \\
		\texttt{images} & The binary image of the kernel and the root
			filesystem \\
		\hline
	\end{tabular}
	\caption{Project directory layout.}
	\label{table:project-dir-layout}
\end{table*}




%%%%%%%%%%%%%%%%%%%%%%%%%%%%%%%%%%%%%%%%%%%%%%%%%%%%%%%%%%%%%%%%%%%%%%%%%%%%%%%%
\section{I2C client driver}
\label{sec:i2c_client_driver}



%%%%%%%%%%%%%%%%%%%%%%%%%%%%%%%%%%%%%%%%%%%%%%%%%%%%%%%%%%%%%%%%%%%%%%%%%%%%%%%%
\section{I2C bus driver}
\label{sec:i2c_bus_driver}

Description of the implementation details of the stn8815 i2c bus driver.



%%%%%%%%%%%%%%%%%%%%%%%%%%%%%%%%%%%%%%%%%%%%%%%%%%%%%%%%%%%%%%%%%%%%%%%%%%%%%%%%
\section{Tools}
\label{sec:tools}

Description of the tools and scripts used throughout the project.



%%%%%%%%%%%%%%%%%%%%%%%%%%%%%%%%%%%%%%%%%%%%%%%%%%%%%%%%%%%%%%%%%%%%%%%%%%%%%%%%
\section{Conclusion}

TODO.



%%%%%%%%%%%%%%%%%%%%%%%%%%%%%%%%%%%%%%%%%%%%%%%%%%%%%%%%%%%%%%%%%%%%%%%%%%%%%%%%
% Add all references without citing
%\nocite{_bibtexkey_}

% We suggest the use of JabRef for editing your bibliography file (Report.bib)
\bibliographystyle{splncs}
\bibliography{Report}

\end{multicols}
\end{document}
